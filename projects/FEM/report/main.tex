\input{../../../preamble.tex}
\usepackage[]{algorithm2e}
\begin{document}
  \maketitle[Scientific Computing III]{Project 2}
  \section{Model Problem}
  With the given model parameters the model equation can be written as
  \begin{align}
    \left(1-\drv[x^2]{^2}\right)T-20 = 0.
  \end{align}
  The solution to this equation is
  \begin{align}
    T(x) = 20+c_1 e^x +c_2 e^{-x}
  \end{align}
  The values of constants $c_1$ and  $c_2$ are given by the boundary values and can be calculated as
  \begin{align}
    T(0) &= 40\nn
    20 + c_1 + c_2 &= 40\nn
    \implies& c_1 = 20-c_2
  \end{align}
  and
  \begin{align}
    T(10) = 200\nn
    20+(20-c_2)e^{10}+c_2 e^{-10} = 200\nn
    c_2\left(e^{-10}-e^{10}\right) = 180-20e^{10}\nn
    c_2 = \frac{180-20e^{10}}{e^{-10}-e^{10}}
  \end{align}
\par For the FEM formulation we will use an equidistant grid and hat-functions, identified by $v$, as base elements. We will also
do a change of coordinates to switch our x-range from $[0,10]$ to the cannonical $[0,1]$.
The Weak Form is given by
\begin{align}
  -\inp{T''}{v} &= \inp{T_\text{ext}}{v}-\inp{T}{v}\nn
  \inp{T'}{v'} + \inp{T}{v}&= \inp{T_\text{ext}}{v}
\end{align}
This means using standard notation we can state the problem as: find $T_h \in V_h$ such that
\begin{align}
  (T_\text{ext}, v_h) = (T_h, v_h) + (T'_h, v'_h)\quad\forall v_h \in V_h
\end{align}
Or, being more explicit:
\begin{align}
  T_\text{ext} \int_0^1 \phi_j(x)\,\d{x} = \sum_{i=1}^{n-1}  c_i \int_0^1  \left(\phi_i(x)\phi_j(x) + \phi_i'(x)\phi_j'(x)\right)\,\d{x}\quad \forall j=1,\ldots,n-1.
\end{align}
\subsection{}
let $\vec c$ be the vector with elements $c_i$, let $\vec b$ be the vector whose $i$'th element is given by $T_\text{ext}\int_0^1\phi_i(x)\,\d{x} = T_\text{ext}2\int_0^h\frac{x}{h}\,\d{x}=T_\text{ext}h$, finally let $K$ be the matrix given by
$K_{i,j} = \int_0^1  \left(\phi_i(x)\phi_j(x) + \phi_i'(x)\phi_j'(x)\right)\,\d{x}$.
Note that
\begin{align}
 \phi_j'(x) &= \frac{1}{h_j}\quad\forall x \in [x_{j-1}, x_j],\nn
            &= -\frac{1}{h_{j+1}}\quad\forall x \in [x_{j}, x_{j+1}],\nn
 &= 0\quad\text{otherwise.}
\end{align}
From this it follows that
\begin{align}
 \int_0^1 \phi_i'(x)\phi_j'(x)\d{x} &= \frac{1}{h_i^2}+\frac{1}{h_{i+1}^2} \quad\text{if }i=j\nn
 &=-\frac{1}{h_i}\quad\text{if }j = i-1\nn
 &=-\frac{1}{h_{i+1}}\quad\text{if }j = i+1\nn
\end{align}
(note that all $h_i$ are considered equal) which, using the integrals given in the workout, implies
\begin{align}
K_{i,i} &= \frac{2}{3}h+\frac{2}{h^2}\\
K_{i,i-1} &= \frac{1}{6}h-\frac{1}{h}\\
K_{i,i+1} &= \frac{1}{6}h-\frac{1}{h}
\end{align}

\end{document}
