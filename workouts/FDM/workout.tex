\documentclass[leqno,a4paper]{article}
\usepackage{hyperref}
\usepackage{caption}
\usepackage[T1]{fontenc}
\usepackage[utf8]{inputenc}
\usepackage{lmodern}
\usepackage[english]{babel}
\linespread{1.25} %easier reading/grading.
\usepackage{amsmath} %d'oh
\usepackage{amsfonts}
\usepackage{graphicx}
\usepackage{bold-extra} %for \mb
\usepackage[margin=2.5cm]{geometry} %for custom margins
\usepackage{enumerate} %for special counters
\usepackage{titlesec} %for section numbering
\usepackage{ifthen}
\renewcommand\thesubsection{\alph{subsection}}
\titleformat{\section}{\it \bf \large}{{\normalfont  \bf \thesection.}}{4pt}{}[]
\titleformat{\subsection}{\it \bf \large}{{\normalfont \bf \quad \large  \thesection \thesubsection)}}{5pt}{}[]
\titleformat{\subsubsection}{\bf \it}{\qquad}{5pt}{}[]

\numberwithin{equation}{section}
\newcommand\norm[1]{\left\lVert#1\right\rVert} %http://tex.stackexchange.com/questions/107186/how-to-write-norm-which-adjusts-its-size
\renewcommand{\O}{\mathcal{O}}
\renewcommand{\bf}{\bfseries}
\renewcommand{\sc}{\scshape}
\renewcommand{\it}{\itshape}
\renewcommand{\div}{\text{div }}
\renewcommand{\Re}{\mathbb{R}}
\newcommand{\op}{\left(}
\newcommand{\cp}{\right)}
\newcommand{\N}{\mathbb{N}}
\newcommand{\mb}{\mathbf}
\newcommand{\nn}{\\\nonumber}
\newcommand{\curl}{\text{curl }}
\newcommand{\inp}[2]{\left<#1, #2\right>}
\renewcommand{\d}[1]{\,\text{d}#1}
\newcommand{\pdrv}[2][x]{\frac{\partial #2}{\partial #1}}
\newcommand{\drv}[2][x]{\frac{\text{d} #2}{\text{d} #1}}
\renewcommand{\maketitle}[2][]{\begin{center}
{{\bf \huge \sc #2\\
\ifthenelse{\equal{#1}{}}{}{{\vspace{-12pt} \Large #1}\\\vspace{5pt}}
\large R.G.A. Deckers}}
\vspace{-1. cm}\rule[2.5 cm]{16 cm}{1 pt}
\vspace{-2.5 cm}
\end{center}}

\usepackage[]{algorithm2e}
\begin{document}
  \maketitle[Scientific Computing III]{Workout 3}

\section{}
 This method can be described (ignoring the boundary conditions) as
\begin{align}
  D_{+,t}u = D_{0,x}u
\end{align}
or equivalently
The error of $D_{+,t}$ can be determined using Taylor expension in the following fashion (we expand around $\bar x$):
\begin{align}
  u(\bar x + \Delta)&\approx
    u(\bar x)\pm \Delta u'(\bar x)
    + \frac{1}{2}\Delta^2u''(\bar x)
    \pm \frac{1}{6}\Delta^3u'''(\bar x)
    + \O(\Delta)^4
\end{align}
From this we find that
\begin{align}
  D_{+,t}u &= \frac{u(x, \bar t + \Delta t) - u(x, \bar t)}{\Delta t},\nn
  &= \frac{\Delta t \pdrv[t]{u} + \frac{1}{2}\Delta t^2\pdrv[t^2]{^2 u} + \O(\Delta t^3)}{\Delta t},\nn
  &= \pdrv[t]{u} + \frac{1}{2}\Delta t\pdrv[t^2]{^2 u} + \O(\Delta t^2).
\end{align}
And
\begin{align}
  D_{0,x}u &= \frac{u(\bar x + \Delta x, t) - u(\bar x - \Delta x, t)}{2\Delta x},\nn
  &= \frac{2 \Delta x \pdrv[x]{u} + 2 \frac{1}{6}\Delta x^3\pdrv[x^3]{^3 u}+\O(\Delta x^4)}{2 \Delta x},\nn
  &= \pdrv[x]{u}+\frac{1}{6}\Delta x^2\pdrv[x^3]{^3 u}+\O(\Delta x^3).
\end{align}
Now we can compute the local truncation error by subtracting the rhs from the lhs of the approximate model (dropping the big-oh terms):
\begin{align}
  \tau(x,t) &= D_{+,t}u-D_{0,x}u\nn
  &= \left(\pdrv[t]{u} + \frac{1}{2}\Delta t\pdrv[t^2]{^2 u}\right)-\left(\pdrv[x]{u}+\frac{1}{6}\Delta x^2\pdrv[x^3]{^3 u}\right)\nn
  &= \pdrv[t]{u} - \pdrv[x]{u} + \left(\frac{1}{2}\Delta t\pdrv[t^2]{^2 u} - +\frac{1}{6}\Delta x^2\pdrv[x^3]{^3 u}\right)
\end{align}
\section{}
The above equation shows that the order-of-accuracy is $\O\left(\Delta t + \Delta x^2\right)$
\section{}
A method is said to be consistent if $\tau(x,t)\to 0$ as $\Delta x, \Delta t \to 0$. This is the case here (the 'real' derivatives are equal
by the definition) so the method is consistent.
\end{document}
